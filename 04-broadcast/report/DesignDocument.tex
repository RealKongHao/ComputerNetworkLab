%        File: DesignDocument.tex
%     Created: 一 3月 26 01:00 下午 2018 C
% Last Change: 一 3月 26 01:00 下午 2018 C
%
\documentclass[UTF8,noindent]{ctexart}
\usepackage[a4paper,left=2.0cm,right=2.0cm,top=2.0cm,bottom=2.0cm]{geometry}
\usepackage{hyperref}
\usepackage{url}
\usepackage{graphicx}
\usepackage{amsmath}
\usepackage{amssymb}
\usepackage{enumitem}
\usepackage{tikz}
\usepackage{float}
\usepackage{xeCJK}
\usepackage{listings}
\usepackage{xcolor}
\lstset{language = c,numbers=left, showstringspaces=false,keywordstyle= \color{ blue!70 },commentstyle=\color{red!50!green!50!blue!50}, frame=shadowbox, rulesepcolor= \color{ red!20!green!20!blue!20 } 
} 
\CTEXsetup[format={\Large\bfseries}]{section}
\usetikzlibrary{graphs}
\title{\CJKfamily{zhkai}计算机网络研讨课实验报告}
\author{{\CJKfamily{zhkai}冯吕}\ $2015K8009929049$}
\date{\today}
\begin{document}
\maketitle
\zihao{5}
\CJKfamily{zhsong}
%\begin{center}
%  \begin{tabular}{|p{15cm}|}
%    \hline
\section*{{\CJKfamily{zhhei}实验题目}} 广播网络实验
%\hline
\section*{{\CJKfamily{zhhei}实验内容}}
在本次实验中,需要构建多节点网络。实验内容主要分为三个部分:
\begin{itemize}
  \item 实现一个节点广播,在该网络结构中,有一个$hub$节点$b1$,另外还有三个主机节点$h1, h2, h3$,每个主机节点均与$b1$相连,$b1$收到每一个数据包以后,需要将该包从其他端口转发出去,从而实现广播的功能。
	\item 使用$iperf$工具测量上面创建的广播网络的链路利用效率。
	\item 自定义一个网络拓扑结构,在该结构中,有三个$hub$节点,分别为$b1,b2,b3$,另外,还有两个主机节点:$h1, h2$,$h1$连接到$b1$,$h2$连接到$b2$。由$h1$向$h2$发送一个数据包,通过抓包观察一个数据包不断被转发的现象。
\end{itemize}

实验需要实现的内容主要有:$main.c$中实现将收到的每一个包从其他端口转发出去;另外,需要写一个$Python$脚本实现上面写到的自定义网络拓扑,进行环路转发实验。

		%\hline
		\section*{{\CJKfamily{zhhei}实验流程}}
		本次实验中,首先需要实现一个广播网络,需要实现$main.c$中的$TODO$部分:将收到的包从其他端口转发出去。由于所有端口都已经存储在$instance->iface\_list$数据结构中,因此,通过宏定义$list\_for\_each\_entry(pos, head, member)$找出所有其他端口,然后通过$iface\_send\_packet$函数将包发送出去即可。实现代码如下:
\begin{lstlisting}
void broadcast_packet(iface_info_t *iface, const char *packet, int len)
{
	iface_info_t *iface_t = NULL;
	list_for_each_entry(iface_t, &instance->iface_list, list)
		if (iface_t->fd != iface->fd)
		  iface_send_packet(iface_t, packet, len);
}
\end{lstlisting}

之后,通过脚本$three\_nodes\_bw.py$创建网络拓扑,然后在$b1$中运行$main.c$编译后的$hub$程序,之后,通过$ping$命令来查看$h1, h2, h3$三个节点两两之间是否能够$ping$通,如果能够$ping$通,则说明实现了节点广播。

节点广播网络构建好之后,需要使用$iperf$工具来测试网络的链路利用率。分两种方式测量,一种方式是$h1$作为$client$,$h2, h3$作为$servers$;另一种方式是$h1$作为$server$,$h2, h3$作为$clients$。\\

在最后一部分实验内容中,需要创建一个环路网络,网络具体结构见\textbf{实验内容}部分。因此,需要完成一个创建该网络拓扑的脚本,脚本内容如下:
\begin{lstlisting}[language = Python]
#!/usr/bin/env python2

from mininet.topo import Topo
from mininet.net import Mininet
from mininet.link import TCLink
from mininet.cli import CLI

# Mininet will assign an IP address for each interface of a node
# automatically, but hub or switch does not need IP address.
def clearIP(n):
    for iface in n.intfList():
        n.cmd('ifconfig %s 0.0.0.0' % (iface))

class BroadcastTopo(Topo):
    def build(self):
        h1 = self.addHost('h1')
        h2 = self.addHost('h2')
        b1 = self.addHost('b1')
        b2 = self.addHost('b2')
        b3 = self.addHost('b3')

        self.addLink(b1, b2, bw=20)
        self.addLink(b1, b3, bw=20)
        self.addLink(b2, b3, bw=20)
        self.addLink(h1, b1, bw=20)
        self.addLink(h2, b2, bw=20)

if __name__ == '__main__':
    topo = BroadcastTopo()
    net = Mininet(topo = topo, link = TCLink, controller = None)

    h1, h2, b1, b2, b3 = net.get('h1', 'h2', 'b1', 'b2', 'b3')
    h1.cmd('ifconfig h1-eth0 10.0.0.1/8')
    h2.cmd('ifconfig h2-eth0 10.0.0.2/8')
    clearIP(b1)
    clearIP(b2)
    clearIP(b3)

    h1.cmd('./disable_offloading.sh')
    h2.cmd('./disable_offloading.sh')

    net.start()
    CLI(net)
    net.stop()
\end{lstlisting}
创建好网络拓扑之后,由$h1$向$h2$发送一个数据包,通过抓包来观察一个数据包不断被转发的现象(使用$wireshark$)来进行观察。
%\hline

\section*{{\CJKfamily{zhhei}实验结果}}
		%\hline
略。
		\section*{{\CJKfamily{zhhei}结果分析}}
			%\hline
		略。
			%  \end{tabular}
			%\end{center}
			\end{document}


